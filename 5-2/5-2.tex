\documentclass{article}
\usepackage[usenames,dvipsnames]{pstricks}
\usepackage{amsfonts}
\usepackage{amsmath}
\usepackage{amssymb}
\usepackage{epsfig}
\usepackage{graphicx}
\usepackage{mathrsfs}
\usepackage{pst-grad} % For gradients
\usepackage{pst-plot} % For axes
\usepackage{subcaption}
\usepackage{tikz}

\title{Notes for Cryptography}
\author{Professor Brian Sittinger}
\date{5/2/16}

\begin{document}
\maketitle
\section{Introduction}
The last homework has been put out. 

\section{Number Field Sieve}
A technique to factor large composite numbers. Given a number $n$ which we wish
to factor chose a degree $d$ depentent on $n$, 
\[ d \approx \sqrt{\frac{\log{n}}{\log{\log{n}}}} \]
in practice. Let 
\[ m = \lfloor \sqrt[d]{n} \rfloor \]
an expand $n$ in base $m$
\[ n = m^{d} + a_{d-1} m^{d-1} + \cdots + a_0 \]
Let $\alpha$ be a root of this (non-rational). See the notes from class. 
Ergo, we may work in $\mathcal{O}(\alpha)$. 

A number is smooth if it factors into small primes. So we find some $\alpha$ in
$\mathcal{O}$ such that $\langle \alpha \rangle$ is smooth in $\mathcal{O}$ and $h(\alpha)$ is
smooth in $\mathbb{Z}$. We have factor bases a list of permissible primes and
units. 

The first list on the notes is of factor bases for $\gcd{(a,b)} = 1$
What we want is 
\[ \alpha_1,\alpha_2, \ldots, \alpha_r \in \mathcal{O} \]
such that
\[ \langle \alpha_1 \alpha_2 \ldots \alpha_r \rangle = \beta^2 \in \mathcal{O} \]

With this, we now use the trick below: if we suppose that
\[ x^2 \equiv y^2 \mod{n}  \]
\[ x \not\equiv \pm y \pmod{n} \]
Then
$\gcd(x-y, n)$ is a factor of $n$. 

And note that
\[ \gcd(1807 - 38, 2501) \equiv 61 \]
Which is prime, and a prime factor if 2501.

\section{Dirchlet's Theorem (Primes in Arithmetic Progression)}
For $m \in \mathbb{N}_{>1}$, let $\gcd(a,m) = 1$. There are infinirely many
primes congruent to $a \pmod{m}$. 

Proof (stretch):
We'll show (the stronger statement) that 
\[ \lim_{s\rightarrow 1^{+}} \frac{\sum_{prime p} p^{-s}}{\sum_{p} p^{-s}} =
1/\phi(m) \]
This is known as the Dirchlet density. A non-zero Dirchlet density implies that
$|A| \Rightarrow \infty$. A corrolary Moreover, there is asymptoticall an equal
number of primes in any congruence class $a \pmod{m}$, where $\gcd(a,m) = 1$. 

Some background material (Fourier Analysis) let $G = {(\mathbb{Z}_{m})}^*$. Define
\[G^* = \{ \chi : G \rightarrow \mathbb{C}^* \} \]
The dual group (set of ``Dirchlet Class'' modulo $m$). A fact:
\[ |G^*| = \phi(m) \]

An example, 
\[ \mathbb{Z}_5^* \]
We use $\langle 2 \rangle$ $\chi$ is completely determined by value sr 2. 2 Has
order $4 = \phi(5)$. 
\[ \chi(2) \in\mathbb{C}^* \]
order divisor of 4. So, $\chi(2) = i^k$ for some $k \in \{1,2,3,4\}$. 

So, now we apply Fourier. We have orthagonality relations, which are usually
represented by $\sin$s and $\cos$s. So what do our orthagonality relations look
like here. 
\[ \sum_{\gcd}\chi(g) = \begin{cases} \phi(m) = |G| &\text{if } \chi = 1 \\
0 &\text{if } \chi \neq 1 \end{cases} \]
\[ \sum_{\chi \in G^*} \chi(g) = \begin{cases} \phi(m) = |G^*| &\text{if } \chi
= 1 \forall\text{ fixed } \chi \in G^* \\ 
0 &\text{if } g \neq e \end{cases}\]
Definition, Fourier Series of $f:G\rightarrow \mathbb{C}$. $s_f(x) =
\sum_{\chi \in G^*}\hat{f}(\chi)\chi(x) $
\[ \hat{f} = \frac{1}{\phi(m)} \sum_{g \in G} f(g) \chi(g^{-1}) \]
$\hat{f} : G^* \rightarrow \mathbb{C}$. The Fourier coefficient (transform of
$f$). I need $s_f(g) = f(g) \forall g \in G $ via orthagolan relation. 

Proof part 1: Let $f: G \rightarrow \mathbb{C}$ be the characteristic function
$f(x) = \begin{cases} 1 & x \equiv a \pmod{m} \\ 0  & x \not\equiv a \pmod{m}
\end{cases}$
Then forall $\chi \in G^*$ we have,
\[ \hat{f}(\chi) = 1/\phi(m) \sum_{x \in G} f(x) \chi(x^{-1}) =
\frac{1}{\phi(m)} \chi(a^{-1}) \]
so, $s_f (x) =  f(x)$ reduces to
\[ \sum_{\chi \in G^*} ( \frac{1}{\phi(a)}\chi(a^{-1}) ) \chi(x) =
\frac{1}{\phi(m)} \sum_{\chi \in G^*}\chi(a^{-1}x) = \begin{cases} 1 & x \equiv
a \pmod{m} \\
0 & \text{else} \end{cases} \]
Now, we consider 
\[ \sum_{p \equiv a \pmod{m}} p^{-s} \]
Key idea: use $f$ to take the sum over all primes $p$ instead of some primes.
Then replace $f$ with its Fourier Series. Then, 
\[ \sum_{p \equiv a \pmod{m}} p^{-s} = \sum_{p} f(p) p^{-s} = \sum_p
\left(\frac{1}{\phi(m)}\sum_x \chi(pa^{-1})\right) p^{-s} =
\frac{1}{\phi(m)}\sum_{\chi} \chi(a^{-1})\sum_p \chi(p)p^{-s} \]
Define $L(z,\chi) = \sum \frac{\chi(n)}{n^s} := \prod_p (1 - \chi(p)p^{-s})^{-1}$. 
At this point we can now use the fundamental theorem of Algebra. Note that
\[ \chi = 1 \Rightarrow L(s,1) = \zeta(s) \]
So, 
\[ \log L(s,\chi) = -\sum_p \log(1-\chi(p)p^{-s}) = \sum_{k = 1} k^{-1}
\chi(p^k) p^{-ks} = \sum_{k = 1} \chi(p)p^{-s} + \sum_{k \geq
2}k^{-1}\chi(p^k)p^{-ks} \]
$\log$ is base $e$. 
We now use the traingle inequality (we know $s,k > 1$). 

\[ < \sum_p \sum_{k = 2}^{\infty} = 1^{-1} 1 p^{-k} \]
\[ = \sum_P{ \frac{1}{p(p-1)} } \]
\[ \leq \sum_{n = 2}^{\infty} \frac{1}{n(n-1)} = 1 \]
By the use of a telescoping sum! 

In summary, 
\[ \sum_{p \equiv a \pmod{m}} p^{-5} = \frac{1}{\phi(m)} \left[ \sum_{\chi}
\chi(a^{-1})\log L(\chi,s) \right] + O(1) \]

We show that the right hand side approaches infinity as $s \rightarrow 1^{+}$.
Facts true for $\chi = 1$. So $\zeta(s)$ has a simple $pk$ at $s = 1$ with
residue 1. 
\[ \zeta(s) = \frac{1}{s-1} + \text{analytic stuff} \] 

For all $\chi \neq 1 $ 
\[ L(1, \chi) \neq 0 \]
and finite. Then
\[ \sum_{p \equiv a(m)} p^{-s} = \frac{1}{\phi(m)} \left[ \sum_{\chi}
\chi(a^{-1}) \log L(s, \chi) \right] + O(1) \]
\[  = \frac{1}{\phi} \left[ \log(\zeta(s)) + \sum_{\chi \neq 1}
\chi(a^{-1})\log(L(\chi,s)) \right] + O(1) \]
$O(1)$ is Big O notation. 
\[ \frac{1}{\phi(m)} \log\left(\frac{1}{s-1}\right) + O(1) \]
Now we apply this back in the original statement, and take the limit. 

There's also something called the natural density. 
\[ \pi_A(x) / \pi(x) \]
Where $\pi$ is the number of primes up to $x$. We can take the limit as $x$
approaches infinity. The reasoning behind this is 
\[ \pi_A(x) = \sum \zeta = \sum f(p)1 \]

\section{Homework}
Part 1b and c are linked - we use part be to do part c. What we're trying to do
is 
\[ \left(\frac{a}{p}\right) = 1 \]
for half of $a \in \{1, \ldots, p-1\}$
\[ a \pmod p \]
So for a given $a$, there are an infinite number of ways to have $p$. Email the
professor if stuck!

\end{document}
