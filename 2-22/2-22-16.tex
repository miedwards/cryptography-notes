\documentclass{article}
\usepackage[usenames,dvipsnames]{pstricks}
\usepackage{amsfonts}
\usepackage{amsmath}
\usepackage{amssymb}
\usepackage{epsfig}
\usepackage{graphicx}
\usepackage{mathrsfs}
\usepackage{pst-grad} % For gradients
\usepackage{pst-plot} % For axes
\usepackage{subcaption}
\usepackage{tikz}

\title{Notes for Cryptography}
\author{Professor Brian Sittinger}
\date{2/22/16}

\begin{document}
\maketitle
\section{Introduction}
We will have our midterm in two weeks. For problem 2 of HW4, the basis should be
fairly obvious, but we need to be clear why it works. The norm should serve to
rule out almost everything. Number 3 is grizzly. The discriminant is not square
free: 
\[ \Delta[1, \sqrt{2},\sqrt{3},\sqrt{6}]  = 2^{10}\cdot 3^2 \]
\[ \alpha = (1/3) (a+b\sqrt{2} +c\sqrt{3}+d\sqrt{6}),a,b,c,d \in {0,1,2} \]
Roughly 82 choices. 
\[ \sigma_1: \sqrt{2} \rightarrow \sqrt{2}, \text{etc}\]
$a = 0$ which reduces this to 21 solutions. This is true since $T(a\alpha)  \in
\mathbb{Z}$. Then we find that $c = 0$ from
\[ \alpha + \sqrt{2}(\alpha) = (1/3)(2c\sqrt{3}) \]
being an algebraic integer of the form $\mathbb{Z}[\sqrt{3}]$. Furthermore we
see that $b$ and $d$ are both 0 since
\[ \alpha + \sigma_3(\alpha) = (1/3)2b\sqrt{2} \]
is an algebraic int in $\mathbb{Z}[\sqrt{2}]$. So we want
\[ \alpha = \frac{d \sqrt{6}}{3} \] is an algebraic integer (in
$\mathbb{Z}[\sqrt{6}]$). Note that the 1/2 case is an algebraic integer. 
New potential basis
\[ \{1,2,3,(1/2)(\sqrt{2}+\sqrt{6}) \} \]
Use MATLAB/python --- we can use some hand waving this time only. We could have
replaced $\sqrt{2}$ instead of $\sqrt{6}$ too. For number 7, they are
essentially the Eisenstein integers, and we should have 6 of them. The new
homework has been posted. 

\section{Quadratic/Cyclotomic Rings}
Quadratic and cyclotomic rings are two sets with non-empty intersect. Some
elements are 
\[ \mathbb{Z}[i], \mathbb{Z}\left[\frac{-1+\sqrt{-3}}{2}\right] \]
which are the Gaussian and Eisensteinian integers. 

\section{Chapter 4 --- Factorization into Irreducibles}
More about units and associates. We define $u$ is a \emph{unit} in ring $R$ iff $u^{-1}
\in R$ or ($\exists v \in R, uv = 1$). Let $x,y$ be two elements in $R$. Then
$x,y$ are \emph{associates} iff 
\[ x = uy \]
for some unit $u \in R$. We denote the \emph{group of units} in $R$ as either
$U(R)$ or $R^*$. For example
\[ \mathbb{Z}^* = \{\pm1\} \]
\[ \mathbb{Z}[i]^* = \{ \pm1, \pm i \]
\[ \mathbb{Z}\left[\frac{-1+\sqrt{-3}}{2}\right]^* = \{ \pm1, \pm \omega, \pm
\omega^2\} \]
\[ K^* = K\setminus \{0\} \]
Where $K$ is a field.

Let $D$ be an integral domain. Integral domains have the zero product property.
Then
\[ x \in D^* \equiv x | 1 \]
and $x$ and $y$ are associates iff $x|y$ and $y|x$. An associate of an
irreducible element is also irreducible. An element is irreducible if factors
are units or associates of itself. For example, in $\mathbb{Z}$, 7 is
irreducible since it only has $\pm 7, \pm 1$ as factors. 

\subsection{Ideals}
Let $x,y \in D$ be non-zero and $D$ be an integral domain. Then ``to contain is
to divide.'' 
\[ \langle x \rangle \supseteq \langle y \rangle \Leftrightarrow x | y \]
We prove this below. Let $x|y$ Then $y = rx$ for some $r \in D$. Let $\langle a
\rangle = \{ ar| r \in R \}$. Thus $\langle x \rangle \supseteq \langle y
\rangle$. Fortunately, all these steps are indeed reversible. Thus this proof is
complete. For example, in $\mathbb{Z}$, consider 2 and 4. Then $2|4$ and
$\langle 2 \rangle \supseteq \langle 4 \rangle $. 

Now is $x$ and $y$ are associates, then 
\[ \langle x \rangle = \langle y \rangle \]
This flows directly from the definition of associates. Now $x \in D^*$ is
equivalent to saying that $\langle x \rangle = D$. Furthermore iff $x$ is
irreducible, then $\langle x \rangle$ is maximal among proper principles ideals
in $D$. An ideal is maximal if there is no ideal between it and the ring $D$.
Check mathworld for details. 

\subsection{Norms}
Theorem: Let $\mathcal{O}$ be the set of integers in $K$. Let $x,y \in
\mathcal{O}$. Then
\[x \in \mathcal{O} \iff N(x) = \pm 1 \]
\[ x|y \implies N(x) | N(y) \]
The proof is something like this: $y = rx$ for some $r \in \mathcal{O}$. So,
$\sigma_i(y) = \sigma_i(rx) = \sigma_i(r) \sigma_i(x)$ for all embeddings
$\sigma_i$. Multiply through $i$ to see that $N(y) = N(r) N(x)$. However, the
converse to this theorem is untrue. Consider $\mathbb{Z}[i] : N(2 \pm i) = 5$.
However, $(2+i) \nmid (2-i)$ and $N(2+i) \mid N(2-i)$. 

If $x,y$ are associates, then $N(x) = \pm N(y)$. This is again, unidirectional,
and it flows from the definition of associates. 

If $N(x)$ is prime, then $x$ is irreducible. Primallity is defined in
$\mathbb{Z}$. Suppose that $x = yz$ for some $y,z\in \mathcal{O}$. We then take
the norm. Then
\[ N(x) = N(y) N(z) \] 
which is a contradiction unless $N(y)$ or $N(z)$ is $\pm 1$. Thus one is a unit,
and the other is an associate. 

\subsection{Existence of Factorizations into Irreducible Elements}
For example, let $\alpha \in B\setminus(B^* \cup \{0\})$, where $B$ is any algebraic integer over
$\mathbb{Q}$. Since $\alpha = {(\sqrt{\alpha})}^2$, and $\sqrt{\alpha} \in B$.,
$\alpha$ is not irreducible. So $B$ has no irreducible elements. This is not the
case in $\mathcal{O}$ for some number field $K$. We define an integral domain is
\emph{Noetherian} iff it satisfies the ``ACC'' that every ascending chain of
ideals is terminates. This is equivalent to saying that every ideal is finitely
generated. $\mathbb{Z}$ is Notherian, since all ideals are principle (1
generator). Consider $\langle 48 \rangle$. Then
\[ \langle 48 \rangle \subset \langle 24 \rangle \subset \langle 12 \rangle \subset
\langle 6 \rangle \subset \langle 2 \rangle \subset \langle 1 \rangle \] 
A hint for the homework is ``missisippi.'' An important note: Noetherian rings
have factors into irreducible elements. We care about this because it turns out
that $\mathcal{O}$ is Noetherian and relates in some way to Dedicand domains.
But we will return to this later. We know that $\mathcal{O}$ is Noetherian
because $(\mathcal{O}, +)$ has an integral basis of degree $n = [k:\mathbb{Q}]$
Therefore free Abelian group of rank $n$. So, any ideal $I \subseteq
\mathcal{O}$ has an integral basis of at most $n$ elements. 

\subsection{Primes vs Irreducible Elements}
Let $D$ be an integral domain. $x \in D_{\neq 0}$ is called \emph{prime} iff
\[ x|ab \implies x|a \text{ or } x|b \]
Note that Primes are irreducible. This is because if $x$ is prime, and $x = ab$,
then $x | ab$ and if $x|a$, $a = xc, c \in D$ then $x = ab  = xcb$ and $1 = cb$
and $b$ is a unit, which makes $x$ irreducible. The converse is false
($\mathbb{Z}[\sqrt{-5}]$). In an integral domain $D$ where there exists
factorization into irreducibles, this factorization is unique iff if each
irreducible is prime. We show this by supposing that each irreducible is
prime. We consider 2 factorizations are identical. Let $x = u_1 p_1 \ldots p_m =
u_2 q_1 \ldots q_m$ where $u_1, u_2 \in D^{*}, p_1,\ldots,p_m,q_1,\ldots q_n$
are primes in $D$. This is trivially true for $m = 0$. We then prove this by
induction. We may cancel out some prime factors. Ergo we see that $m = n$
and that the factorization is unique. Now we assume unique factorization into
irreducibles, and show that each irreducible is prime. Suppose $p\in D$ is
irreducible, suppose $p|ab$ and $a,b$ is nonzero. So $ab - pc$ for some $c \in
D$ Factor into irreducibles, 
\[ p (u_1, r_1\ldots r_s) = (u_2 p_1\ldots p_m)(u_3 q_1 \ldots q_n) \]
At this point we apply the unique factorization as we assumed. Then $p$ is an
associate to one of $p_i, q_j$. Therefore, $p \mid a$ or $p \mid b$
respectively. 

Consider this example (cultural note), how many quadratic number rings case with
unique factorization:
Gaussians, Eisensteinians, $d = -1,-2,-3,-7,-11,-19,-43,-67,-163$ and that's
all! Gaussian is $d = -1$ and Eisensteinian is  $d = -3$. In the Real numbers,
this list is unknown. For example, all square free integers under $50$ except
10, 15, 26, 30, 34, 35, 39, 42. The first positive prime $d$ for which we lose
factorization is $79$. Moving onto the cyclotomics, the first root of unity for
which we lose unique factorizations if $23$. 

We now consider Euclidean Domains, which is defined as an integral domain for
which there exists $\phi$ such that
\[ \phi: D_{\neq 0} \rightarrow \mathbb{N} \]
such that if $a|b$, $a,b \in D_{\neq 0}$, then $\phi(a) \leq \phi(b)$.
Furthermore for $a,b \in D_{\neq 0}$ there exists $g,r \in D$ such that
\[ a = bq + r \]
where $r = 0$ or $\phi(r) \leq \phi(b)$. For example, $\mathbb{Z}$ is Euclidean
with $\phi(n) = |n|$. Also $k[x]$ is Euclidean with $\phi(p(x)) =
\text{deg}(p(x))$ where $k$ is a field. Euclidean domains are principle ideal
domains. That is, given two elements in the idea, we may use the Euclidean
algorithm to find the principle generator. And so, as a principle ideal domain,
we are also in a unique factorization domain. Consider the Imaginary Quadratic
case. This is euclidean iff $d = -1, -2, -4, -7, -11$ with $\phi(\alpha) =
N(\alpha)$. Corollary: If $d = -19, -43, -67, -163$, then
$\mathcal{O}_{\mathbb{Q}(\sqrt{d})}$ but is still a UFD\@. We show this by showing
that $\phi$ satisfies some of the required properties. $a|b \implies N(a) <
N(B)$ and that for any $\epsilon \in \mathbb{Q}(\sqrt{d})$,  there exists
$\kappa \in \mathcal{O}$ such that $N(\epsilon - \kappa) < 1 $ which is
equivalent to the second rule. Suppose $\epsilon = r + s \sqrt{d}$ where $d <
0, r,s \in \mathbb{Q}$. Case 1: $d \neq 1 \pmod{4}$. For $\kappa$ we take $x,y\in
\mathbb{Z}$ as the integers closest to $r,s$ respectively. $|x-r|,|y-s| \leq
1/2$. So we let $\kappa = x + y \sqrt{d}$. From this we calculate the norm, and
see that 
\[ N(\kappa - \epsilon) = 3/4 < 1 \]
as required. Case 2: 
\[ d = 1 \pmod{4} \]
We let 
\[ \kappa = x + y\left( \frac{1 + \sqrt{d}}{2} \right), x,y\in \mathbb{Z}\]
So we find
\[ N(\kappa - \epsilon) = (r - x - y/2)^2 + |d|(s - y/2)^2 \]
Pick $y$ closest to $2s$, or $|2s - y| < 1/2$ and then find $x \in \mathbb{Z}$
such that $|r -x - y/2| < 1/2$.
And so we see that
\[ N(\kappa - \epsilon) < (1/2)^2 + \frac{11}{4}\left(\frac{1}{2}\right)^2 =
15/16 < 1 \]

We now consider an example of factoring in $\mathbb{Z}[i]$. We factor $4+7i$.
First we use the Norm, $N(4+7i) = 5 \cdot 13 $. Note that $2$ is no longer
prime: $2 = (1+i)(1-i)$.  Note that
\[ p \equiv 3 \pmod{4} \]
are still prime in $\mathbb{Z}[i], N(p) = p^2$. However 
\[ p \equiv 1 \pmod{4} \]
can be factored. Primes with norm 5, $5 = (2+i)(2-i)$. We must try all factors
(using trial and error) to find the one that works. Note that the factorization
is not unique, but that is not important.
\[ \frac{4+7i}{2+i} = 3 + 2i \]
\[ N(5) N(13) \equiv 1 \pmod{4} \]
so we know that that we have the prime factors $(2+i)(3+2i)$


\end{document}
