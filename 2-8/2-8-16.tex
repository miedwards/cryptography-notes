\documentclass{article}
\usepackage[usenames,dvipsnames]{pstricks}
\usepackage{amsfonts}
\usepackage{amsmath}
\usepackage{amssymb}
\usepackage{epsfig}
\usepackage{graphicx}
\usepackage{mathrsfs}
\usepackage{pst-grad} % For gradients
\usepackage{pst-plot} % For axes
\usepackage{subcaption}
\usepackage{tikz}

\title{Notes for Cryptography}
\author{Professor Brian Sittinger}
\date{2/8/16}

\begin{document}
\maketitle
\section{HW2 Questions}

See HW2.tex. Also, hw3 is supposedly harder.

\section{Chapter 2}
We're going through half of chapter 2. I should catch up on my reading. 
\subsection{Field Extensions}
A Field is a commutative ring with multiplicative inverses. Let $k$ be a field.
Then $L$ is a field extension of $k$ iff $L$ is a field containing $k$. For this
course $k = \mathbb{Q}$ most of the time. For example $\mathbb{C}$ is a field
extension of $\mathbb{R}$. $\mathbb{R}$ is a field extension of $\mathbb{Q}$.
And $\mathbb{Q}(i) = \{ a + bi | a,b \in \mathbb{Q} \}$ is a field extension of
$\mathbb{Q}$. When $L$ is a field extention of $k$, we write $L:k$. $L$ is a
Vector Space over $k$. So, we can discuss the ``Dimension'' called ``the
degree'' of the field extension over $k$ written $[L:k]$. 
\subsubsection{Examples}
\[ [\mathbb{Q}(i) : \mathbb{Q}] = 2 \]
because $\mathbb{Q}(i)$ has basis $\{1,i\}$ over $\mathbb{Q}$. 

\subsection{Tower Law}
Let 
\[ K \subseteq L \subseteq M \]
be a ``tower'' of 3 fields, or 3 fields such that
\[ L : K, M : L \]
Then
\[ [ M : K ] = [ M : L] \cdot [ L : K ] \]
This is used in homework 3. 

Suppose we wanted to find
\[ [ \mathbb{Q}(\sqrt{2}, \sqrt{3}) : \mathbb{Q} ] \]
we may now consider
\[ [ \mathbb{Q}(\sqrt{2}) : \mathbb{Q}] = 2\]
and 
\[ [ \mathbb{Q}(\sqrt{2},\sqrt{3}) : \mathbb{Q}(\sqrt{2}) ] = 2\]
Note: The minimal polynomial gives the degree of the extension. Be thorough. 

Generally, we will concern ourselves with finite extensions, due to relevance.
We define 
\[ L : K, \alpha \in L \]
\begin{enumerate}
\item If there exists a polynomial $p(t) \in K[t]$ such that $p(\alpha) = 0$.
Then $\alpha$ is algebraic over the base field $K$. 
\item Otherwise, $\alpha$ is not algebraic but trancendental.
\end{enumerate}
We'll let $K = \mathbb{Q}$ in this course. Examples of trancedental numbers are
$\pi$ and $e$. Consider
\begin{verbatim}https://en.wikipedia.org/wiki/Liouville_number\end{verbatim}
So how many algebraic numbers and trancendental numbers over $\mathbb{Q}$ exist?
There are countably many algebraic numbers, and uncountably many trancendentals.
We can see this by first noting that $\mathbb{Q}$ is countable, and then noting
that $\mathbb{C}$ is uncountable. 

Suppose $\alpha$ is algebraic over $K$. Then the monic polynomial $p(K)$ of smallest
degree for which $p(\alpha) = 0$ is called the minimal polynomial of $\alpha$
over $K$. 

Some interesting notes:
\begin{itemize}
\item The minimal polynomial is irreducible over $K$. 
\item Take $k = \mathbb{Q}$. Then we can clear ``denominators'' so that we have
a minimal polynomial with coefficients in the integers
\end{itemize}

Let $L:K$ with $\alpha \in L$. Then $\alpha$ is algebraic over $K$ is equivalent
to saying that
\[ [ K(\alpha) : K ] < \infty \]
Moreover,
\[ [ K(\alpha) : K ] = \text{deg}(\text{min poly of }a) \]
and 
\[ K(a) = K[a] \]
the difference in notations refers to the difference between a ring and a field.
That is
\[ K[,a_1, a_2, \ldots, a_n] \]
is the smallest ring containing $a_1,\ldots,a_n$. This is in essence a set of
polynomials in $a_1, \ldots, a_n$ with coefficients in $K$. Similarly, 
\[ K(a_1, \ldots, a_n) \]
is the smallest field containing $a_1,\ldots,a_n$. It is the set of rational
functions in $a_1, \ldots, a_n$ with coefficients in $K$. What are rational
functions? Similar to polynomials, it can be defined
\[ k(a_1,\ldots,a_n) = \left\{\frac{f(a_1, \ldots, a_n)}{g(a_1, \ldots,
a_n)} \mid f,g \in K[a_1,\ldots,a_n] \text{ and } g \neq 0 \right\} \]

Let $d$ be a squarefree integer. Look at 
\[ \mathbb{Q}[\sqrt{d}] \]
Then
\[ [ Q\mathbb{Q}[\sqrt{d}] : \mathbb{Q} ] = 2 \]
because $x^2 = d$ is the minimum polynomial of $\sqrt{d}$ over $\mathbb{Q}$.
And $\sqrt{d}$ is algebraic over $\mathbb{Q}$. Also, 
\[ \mathbb{Q}[\sqrt{d}] = \left\{ a \cdot 1 + b \sqrt{d} \mid a,b \in \mathbb{Q}
\right\} \]
has basis $\{1, \sqrt{d}\}$. This is closed under addition and multiplication.
Next, we note that
\[ \mathbb{Q}(\sqrt{d}) = \mathbb{Q}[\sqrt{d}] \]
We show that the field is contained within the ring by using the conjugate, that
is
\[ \frac{1}{a _ b \sqrt{d}} \cdot \frac{ a - b\sqrt{d} }{a - b\sqrt{d} } \]
Now, $\mathbb{Q}[\sqrt{d}]$ is a quadratic number field. Look this up in more
detail. Back to algebraic numbers and integers!

\section{Algebraic Numbers and Integers}
We define the set of all algebraic numbers over $\mathbb{Q}$ sometimes denoted
$\mathbb{A}$ or $\bar{\mathbb{Q}}$. In the homework, we may show that
\[ [ \mathbb{A} : \mathbb{Q} ] = \infty!! \]
where we count down by two for double factorial. 

Note that $\mathbb{A}$ is a subfield of $\mathbb{C}$. Let $\alpha, \beta \in
\mathbb{A}$. Then by the tower law
\[ [p\mathbb{Q}(\alpha, \beta)] =
[\mathbb{Q}(\alpha,\beta):\mathbb{Q}(\alpha)] \cdot [\mathbb{Q}(\alpha) :
\mathbb{Q}] \]
We are multiplying two finite extensions together, since $\beta$ is algebraic
over $\mathbb{Q}$ and so $\mathbb{Q}(\alpha)$ is too. Therefore
$\mathbb{Q}(\alpha, \beta)$ is a finite extension over $\mathbb{Q}$ and so
closure and inverses under addition and multiplication are guaranteed.

Now we define $K$ as an algebraic number field over $\mathbb{Q}$ if $K$ is a
subfield of $\mathbb{C}$ and $[K,\mathbb{Q}]$ is finite. 
So, $K$ is a subfield of $\mathbb{A}$ and $K = \mathbb{Q}(a_1,\ldots,a_n)$ for
some $a_1,\ldots,a_n \in K$. We can do better, though --- we only need 1
generator! That is $K = \mathbb{Q}(\theta)$ for some $\theta \in \mathbb{A}$.
See Theorem 2.2 from the book. 

To find the professor's mailbox, look for a copy room past Chris, a guy with a
computer. 

Algebraic integers over the rationals are a zero of a monic polynomial in the integers.

\begin{tabular}{c|c}
Number Fields & Number Rings \\
$\bar{\mathbb{Q}}$ or $\mathbb{A}$ & $\mathbb{B}$ \\
$\mathbb{Q}(\theta) = K$ & $\mathbb{Z}[\alpha] = \mathcal{O}$ \\
$\mathbb{Q}$ & $\mathbb{Z}$ \\
Number Field Tower & Number Ring Tower\\
\end{tabular}

For example, if $k = \mathbb{Q}(i)$, then $\mathcal{O} = \mathbb{Z}[i]$.

\section{Conjugates, norms and traces}
This has a flavor of Galois theory to it. Theorem 2.4 from the book. 
Let $k = \mathbb{Q}(\theta)$ be a number field of degree $n$ over $\mathbb{Q}$.
Then there are exactly $n$ distinct embeddings $\sigma_j: K \rightarrow
\mathbb{C}$ with $j = 1, \ldots, n$. Embeddings are one to one ring
homomorphisms that fix $\mathbb{Q}$. That is $\sigma_j(q) = q \forall q \in
\mathbb{Q}$. Moreover, the elements $\sigma_j(\theta)\equiv \theta_j$ for each
$j = 1,\ldots,n$ are the distinct zeros in $\mathbb{C}$ of the minimal
polynomials of $\theta$ over $\mathbb{Q}$. The elements $\theta_1, \ldots,
\theta_n$ are the $k$-conjugates of $\theta$.

\subsection{Example}
More about $K=\mathbb{Q}(\sqrt{d})$ where $d$ is square free. Recall that the
minimal polynomial is $x^2 - d$ with zeroes $\pm \sqrt{d}$. By our theorem,
there exists $ [\mathbb{Q}(\sqrt{(d)}) : \mathbb{Q} ] = 2$ embeddings. And so
\[ \sigma_1(1) = 1 \text{ and } \sigma_1(\sqrt{d}) = \sqrt{d} \]
So 
\[ \sigma_1(a + b \sqrt{d}) = a + b \sqrt{d} \]
So we see that this is the identity map. 

Let us now consider $\sigma_2$ where
\[ \sigma_2(1) = 1 \text{ and } \sigma_2(\sqrt{d}) = - \sqrt{d} \]
And so $\sigma_2$ is what we used to call ``conjugation.''

\subsection{Another Example}
Let $K = \mathbb{Q}(6^{1/3})$. The minimum polynomial is $x^3 - 6$ with three
roots $\sqrt{6}, \omega 6^{1/3}, \omega^2 6^{1/3}$
where $\omega = e^{\frac{2 \pi}{3}}$ is a cube root of unity. By our theorem,
there are 3 embeddings. 
\begin{enumerate}
\item $\sigma_1( 1 ) = 1$
\item $\sigma_1( 6^{1/3} ) = 6^{1/3}$
\item $\sigma_1(6^{2/3}) = 6^{2/3}$ Because it's a homomorphism. 
\end{enumerate}
And so $\sigma_1$ is the identity map. 

\[ \sigma_2(1) = 1, \sigma_2(6^{1/3}) = \omega 6^{1/3}, \sigma_2(6^{1/3} )=
(\omega 6^{1/3})^2 \]
\[ \sigma_3(a+b6^{1/3}+c 6^{2/3}) = a + b(\omega^2 6^{1/3}) + c(\omega^2
6^{1/3})^2 \]

Let $K = \mathbb{Q}(\theta)$ of degree $n$ over $\mathbb{Q}$, and
$\{a_1,\ldots,a_n\}$ be a basis of $K$ over $\mathbb{Q}$. We define the
``discriminent'' 
\[ \Delta[a_1,\ldots,a_n] = \{ \det(\sigma_i(a_j)) \}^2 \]

We define the Trace of $a$ denoted $\text{Tr}(a)$ as
\[ \text{Tr}(a) = \sum_{j=1}^n = \sigma_j(a) \]

We define the Norm for $a \in K$ as
\[ N(a) = \prod_{j=1}^n \sigma_j (a) \]
And so we note that $\Delta, \text{Tr}, N \in \mathbb{Q}$. Moreover, if we
replace $K$ with $\mathcal{O}_k$, then $\Delta, \text{Tr}, N \in \mathbb{Z}$.

Some other facts. The Trace is additive and the Norm is multiplicative. 

We return now to $K = \mathbb{Q}[\sqrt{d}]$. 
\[ \text{Tr}(a + b \sqrt{d}) = 2a \]
\[ N(a + b \sqrt{d}) = a^2 -db^2 \]
\[ \Delta[1, \sqrt{d}] = \begin{vmatrix} 1 & \sqrt{d} \\ 1 & -\sqrt{d}
\end{vmatrix}^2 = 4d \]

\end{document}
