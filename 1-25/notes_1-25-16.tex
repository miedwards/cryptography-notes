\documentclass{article}

\usepackage{amsfonts}
\usepackage{amsmath}


\title{Notes for Cryptography}
\author{Professor Brian Sittinger}
\date{}

\begin{document}
\maketitle
\section{Textbooks}
Homework is DUE NEXT MODNDAY
Supplemental Books:\\
Daniel Marcus, Number Fields. Jurgan Neukirch, Algorithmic Number Theory. \\
But the primary textbook is ``Algebraic Number Theory and Fermat's Last Theorem'' by 
Ian Stewart and David Tall 4th edition though the edition isn't important. 

\section{Office Hours}
Officially 12--1pm. But actually from around 10:45--1:30pm. Another break is in
his schedule after DST around 2:45pm. 

\section{Assignments}
There will be two take home exams. There will be a final exam and a final
project. In extenuating circumstances email the professor. Around the 6th and
12th weeks we will have midterms. One will be around easter break.
\texttt{http://faculty.csuci.edu/brian.sittinger} Has all the info. First HW
is up tomorrow morning. 

\section{Introductions}
Ian - highschool.
Math95
Su persoky teaching math 94
Mohita
Alex MC beurm First year math
Alex - 2 years - teaches
Michel luis - 3rd semester
Merina math95 last semester
Mira - last semester math 95
Moorpark Highschool - 3/4th semester
Dana - teaches precalc 
Vicne Fergusson math 95 - 3rd semester
Maria - teaching math 95
Jennifer Lu Not teaching math 
Ivan first semester math 95
Matt CS, Softwaree
Cara
David second semseter - 
Vijay CS - 3rd sem
Dhruv 
Mark
George Calc 1
Kevin 

\section{Crash Course Review of Elementory Number Theory}
Number theory is sometimes called Arithmetic. We then go into the basic sets.
Number theory is the study of the proprertyies of $\mathbb{Z}$ or
($\mathbb{N}$). The multiplicative structure of $\mathbb{Z}$ is of main focus to
us. We say that $a|b$ ($a$ divides $b$) iff there exists $c \in \mathbb{Z}$ such
that $ a = bc$. Now let $a,b,c \in \mathbb{Z}$. Then
\begin{itemize}
\item $a|b \Rightarrow a | bc$
\item $a|b, a|c \Rightarrow a|(b+c), a|bc$
More generally 
\[ a| mb + nc \forall n,m \in \mathbb{Z} \]
\item $1 | a, a | 0$
\item A Prime or an irreducible number is one that is only divided by 1 and
itself. A prime will have exactly 2 prime factors - so 1 is not prime. An
integer not equal to $\pm 1$ with more than 2 prime factors is composite.
\end{itemize}
The fundamental theorem of Arithmetic. Every integer not in $\{0,-1,1\}$ has a
unique prime factorization (up to order and signs). 

We define the GCD (Greatest Common Divisor) as the largest positive integer
which divides all the arguments. LCM is the least common multiple, is the
smallest positive integer which is a multiple of all arguemnts. For the GCD we
use the minimum of each power of a factors and the LCM we use the maximum for
each power of factors. If the GCD of two numbers is 1, we say that they are
coprime or relatively prime. If any subset of the integers have the property
that any two of them are relatively prime, that subset is ``pairwise relatively
prime.'' 3, 5, and 9 are relatively prime, but not pairwise relatively prime. 

Some properties: 
\begin{itemize}
\item if $a|b, \text{ and } c|b$ then, $\gcd(a,c)|b$.
\item Let $p$ be a prime. Then if $p|ab$ Then
\[ p | a \text{ or } p | b \]
This is an alternative definition for primality. 
\end{itemize}

\section{Modular Arithmetic}
Fix $m \in \mathbb{N}_{>1}$. Given $a,b \in \mathbb{Z}$ we write 
\[ a \equiv b \pmod m \] 
to say that
\[ m | a - b \]
That is $a$ and $b$ have the same remainder when divided by $m$. And if 
\[ a \equiv 0 \pmod m \]
$m | a$ since there exists some $k \in \mathbb{Z}$ such that
\[ a = 0 + km \]
\subsection{Solving Congruences}
Addition, and multiplication over a modulus is simple. Division is possible when 
$m$ is prime. Then $\mathbb{Z}_p$ is a field. If 
\[ ac \equiv dc \pmod{m} \]
Then 
\[ a \equiv d \pmod{\frac{m}{\gcd{m,c}}} \]
However, $4$ has no inverse mod 12 and
\[ 4x \equiv 8 \pmod{12} \]
has the solution
\[ x \equiv 2 \pmod 3 \]
or 
\[ x \equiv 2, 5, 8, 11 \pmod{12} \]

A new example. 
\[ x^2 \equiv 5 \pmod{11} \]
Note that 
\[ 5 \equiv 16 \pmod{11} \]
And since 11 is prime, this has exactly 2 solutions: $\pm 4$. This can be
rewritten as 
\[ 11 | (x+4)(x-4) \] as expected.
Also, remember that
\[ x^\equiv 6 \pmod{11} \] 
Has no solution since
\[ 0^2, (\pm1)^2, (\pm 3)^2, (\pm 4)^2, (\pm 5)^2 \]
goes up to 5. 

\subsection{Diophantine Equations}
Consider
\[ x^2 + y^2 = 4027 \]
It has no integer solutions. Consider this mod 4. 
\[ x^2 + y^2 \equiv 3 \pmod 4 \]
However, this has no solution since 
\[ 0, (\pm 1), (\pm 2) \]
fails.  

\subsection{Fermat's Little Theorem}

If the GCD of $a$ and $p$ is 1, and $p$ is prime, then 
\[ a^{p-1} = 1 \pmod{p} \]
This is useful for reducing large exponents to something managable. For example
\[ 3^{105} \mod 13 \]
is equivalent to
\[ \left(3^{13-1}\right)^{8} \cdot 3^9 \equiv 1 \pmod{13} \]

\subsubsection{Euler Phi Function}
Let $\phi(n)$ be the number of integers from 1 to $n$ inclusive that are
relatively prime to $n$. Then if 
\[ \gcd(a,m) = 1, a^{\phi(m)} \equiv 1 \mod m \]
And we get Fermet's Little theorem again when $m$ is prime. 

Another example
\[ 47^42 \pmod{100} \]
47 and 100 are coprime, so
\[ \phi(100) = \frac{\phi(2^2)}{2}\cdot\frac{\phi(5^2)}{20} \]

\subsection{Wilson's Theorem}
$p$ is prime is equivalent to
\[ (p-1)! \equiv -1 \pmod{p} \]

\subsection{Chinese Remainder Theorem}
Suppose that 
$m_1, \ldots, m_k$ are pairwise relatively prime. Then 
$x \equiv a_i \mod m_i$
for all $i \in [1,k]$ has a unique solution mod $m_1 m_2 \ldots m_k$. Or 
\[ \mathbb{Z}_{m_1\ldots m_k} \cong \mathbb{Z}_{m_1} \times \cdots \times
\mathbb{Z}_{m_k} \]

An example,
\[ x \equiv 1 \mod 2 \]
\[ x \equiv 2 \mod 3 \]
\[ x \equiv 3 \mod 11 \]
Then
\[ x \equiv 5 \mod 6 \]
\[ x \equiv 3 \mod 11 \]
So
\[ 5 + 6k = 3 \mod 11 \]
\[ k = 7 \mod 11 \]

\end{document}
