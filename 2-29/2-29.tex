\documentclass{article}
\usepackage[usenames,dvipsnames]{pstricks}
\usepackage{amsfonts}
\usepackage{amsmath}
\usepackage{amssymb}
\usepackage{epsfig}
\usepackage{mathtools}
\usepackage{graphicx}
\usepackage{mathrsfs}
\usepackage{pst-grad} % For gradients
\usepackage{pst-plot} % For axes
\usepackage{subcaption}
\usepackage{tikz}

\title{Notes for Cryptography}
\author{Professor Brian Sittinger}
\date{}

\begin{document}
\maketitle
\section{Introduction}
HW problem 6 
\[ \mathbb{Q}_{P} = \{a/b | b \nmid a, a,b \in \mathbb{Z}, b \neq 0 \} \]
this is a subset of $\mathbb{Q}$. A:
\[ (\mathbb{Q}_{(p)})^* = \{ \frac{a}{b}| p\nmid a,b \} \]
B: Only irreducible in $\mathbb{Q}_{(p)}$  come from factors whose num are
divisible by $p$. Therefore 
\[ p^k m / n , p \nmid m, n \]
\[ k = 1 \]
for irreducible $s$. 

\section{Chapter 5 --- Factorization into Prime Ideals}
\[ 6 = 2 \cdot 3 = (1+\sqrt{5})(1-\sqrt{5}) \]
Is a non-unique factorization into irreducibles. To restore factors, we will
pass to prime ideals. 

Let $I$ be a proper ideal in a ring $R$. Then we define $I$ as a \emph{prime
ideal} iff whenever 
\[ JK \subseteq I \]
for some ideals $J,K \subseteq R$, we have $ J \subseteq I$ or $K \subseteq I$.
Furthermore, we define $I$ to be \emph{maximal} iff there are no ideals strictly
between $I$ and $R$. 

If $R$ is an integral domain (no zero divisors and $ab = 0 \implies a = 0$ or $b
= 0$) then 
\[ \langle p \rangle \text{ is a prime ideal} \iff p \text{ is prime or } p = 0
\]
The only ideals in a field $F$ are $\{0\}$ or $\langle 1 \rangle = F$. 
Lemma: let $I$ be an ideal of $R$. Then $I$ is maximal iff $R/I$ is a field.
Also, iff $I$ is prime, $R/I$ is an integral domain. A corollary of this is that
maximal ideals are prime. The converse is not necessarily true. For example
$\langle x \rangle$ is prime in $\mathbb{Z}[x]$ but not maximal. Note that
\[ \mathbb{Z}[x] / \langle x \rangle \cong \mathbb{Z} \]

We may now approach the following theorem. $\mathcal{O}$ is a ``Dedicand
Domain.'' That is, $\mathcal{O}$ is an integral domain with field of fractions
$K$. One example of this is $\mathbb{Z}$ and $\mathbb{Q}$. Furthermore,
$\mathcal{O}$ is Noetherian. Additionally, if $\alpha \in K$ satisfies a monic
polynomial $f(x)\in \mathcal{O}[x]$ (with coeficients in $\mathcal{O}$) then
$\alpha \in \mathcal{O}$. That is $\mathcal{O}$ is ``integrally closed.'' For example
\[ \mathbb{Z} [-\sqrt{3}] \neq \mathcal{O} \text{ but }
\mathbb{Z}\left[\frac{-1+\sqrt{-3}}{2}\right]=\mathcal{O}_k \]
Every nonzero prime ideal of $\mathcal{O}$ is maximal (within Dedicand Domains).
This final statement is basicailly theorem 2.10 from the book. I prove it below:
let $\mathfrak{p}$ be a prime ideal of $\mathcal{O}$. Let $\alpha \in
\mathfrak{p}$ be nonzero. Let $n = [K:\mathbb{Q}]$. So $N := N(\alpha) =
\alpha_1(\alpha_2\ldots\alpha_n) \in \mathfrak{p}$ where $a_i, 1\leq i \leq n$
are the conjugates of $n$. So $\langle n \rangle \subseteq \mathfrak{p}$ Then 
$\mathcal{O}/\mathfrak{p}$ is finite because $\mathcal{O}/\langle p \rangle$
is finite, because $\mathcal{O} / \langle N \rangle$ is finite of order $N^n$
and $\langle N \rangle \subseteq \mathfrak{p}$. Moreover
$\mathcal{O}/\mathfrak{p}$ is an integral domain. Therefore
$\mathcal{O}/\mathfrak{p}$ is a field (because finite integral domains are
fields). And so we see that $\mathfrak{p}$ is maximal as required. 

Some reminders:
\begin{enumerate}
\item $\mathfrak{a}|\mathfrak{b} \iff \mathfrak{b} \subseteq \mathfrak{a}$,
where $\mathfrak{a}$ and $\mathfrak{b}$ are ideals. 
\item $\mathfrak{a} + \mathfrak{b} = \{ a + b | a \in \mathfrak{a}, b \in
\mathfrak{b} \}$
\item $\mathfrak{a}\mathfrak{b} = \{\sum_{k=1}^{n} a_kb_k | a_k\in \mathfrak{a}, b_k \in \mathfrak{b}
\}, n \in \mathbb{N}$. In the finitely generated case, 
\[ \mathfrak{a} = \langle a_1, a_2, \ldots, a_j \rangle \]
\[ \mathfrak{b} = \langle b_1, \ldots, b_l \rangle \]
Then $\mathfrak{a}\mathfrak{b} = \langle a_i b_k | i = 1,\ldots,j, k =
1,\ldots,l \rangle$. 
\end{enumerate}
We now move onto our unique factorization theorem. Every ideal $\mathfrak{a} \in
\mathcal{O}$ different from $\langle 0 \rangle, \langle 1 \rangle$
admits a unique factorization $\mathfrak{a} =
\mathfrak{p}_1\ldots\mathfrak{p}_r$ into prime ideals
$\mathfrak{p}_1,\ldots,\mathfrak{p}_r \subseteq \mathcal{O}$ (see pages
107-110). Which is unique up to order of the factors. The proof is in the book,
but I give a sketch here. Let $\mathfrak{a} \subseteq \mathcal{O} \neq
\langle0\rangle, \langle 1 \rangle$. Then there exist prime ideals in
$\mathcal{O}$ such that $\mathfrak{p}_i \subseteq \mathfrak{a}$. For an ideal
$\mathfrak{a} \subseteq \mathcal{O}$ we define $\mathfrak{a}^{-1} = \{ x \in K |
x\mathfrak{a} \subseteq \mathcal{O} \}$. This is called a ``Fractional Ideal.''
Check $\mathfrak{a}\mathfrak{a}^{-1} = \mathfrak{a}^{-1}\mathfrak{a} = \langle 1 \rangle =
0$. If
\[ \mathfrak{a} \not \subseteq \mathcal{O} \]
Then 
\[ \mathfrak{a}^{-1} \not\supseteq \mathcal{O} \]
If $\mathfrak{a} \neq \{ 0 \}$ and $\mathfrak{a} S \subseteq \mathfrak{a}$ for
any subset $S$ of $K$. Then $S \subseteq \mathcal{O}$. $\mathfrak{p}$ maximal,
$\mathfrak{p}\mathfrak{p}^{-1} = \mathcal{O}$. For all nonzero $\mathfrak{a}$,
$\mathfrak{a}\mathfrak{a}^{-1} \subseteq \mathcal{O}$. Define the fractional
ideal of $K$ as follows. A finitely generated $\mathcal{O}$ submodule
$\mathfrak{A} \neq \{0\}$ of $K$ iff there does not exist non-zero $c \in
\mathcal{O}$ such that $c\mathfrak{a} \subseteq \mathcal{O}$ is an ideal of
$\mathcal{O}$. Every fractional ideal $\mathfrak{a}$ has an inverse
$\mathfrak{a}^{-1}$ such that $\mathfrak{a}\mathfrak{a}^{-1} = \mathcal{O}$.
Note that $J$ is the set of all fractional ideals of $K$ is an Abelian group. 
Every nonzero $\mathfrak{a}$ is a product of prime ideals. Thus, the prime
factorization is unique. Furthermore, fractional ideals have unique
factorization. 

An example: $\mathbb{Z}[\sqrt{-5}]$ revisited. Let $\mathfrak{p} = \langle 2, 1
+ \sqrt{-5} \rangle, \mathfrak{q} = \langle 3, 1 + \sqrt{-5} \rangle, 
\mathfrak{r} = \langle 3, 1 - \sqrt{-5} \rangle$
We claim that the ideals are maximal and so prime. Note that for $\mathfrak{p}$
\[ \left| \mathbb{Z}[\sqrt{-5}]/\langle 2 \rangle \right| = 4 \]
Since $\langle 2 \rangle\not\subseteq \langle 2, 1 + \sqrt{-5} \rangle$
\[ \left|\mathbb{Z}[\sqrt{-5}]/\mathfrak{p}\right| \]
has order \emph{dividing} 4 by Lagrange theorem. The only possibility for 
\[ \left|\mathbb{Z}[\sqrt{-5}]/\mathfrak{p}\right| \]
is 2 because
\[ \langle 2 \rangle \not \subseteq \mathfrak{p} \text{ and can't be }
\mathbb{Z}[\sqrt{-5}] \]
Therefore, $\mathbb{Z}[\sqrt{-5}] / \mathfrak{p} \cong \mathbb{Z}_2$, is a
field. In other words, a ring with $p$ elements with prime $p$ is a field, and
so that $\mathfrak{p}$ is maximal since $\mathbb{Z}[\sqrt{-5}]$ is a dedicand
domain. 

Our second claim is that these ideals are \emph{not} principle. Suppose that
$\mathfrak{p}$ are not principle. Suppose that $\mathfrak{p} = \langle a + b
\sqrt{-5} \rangle $, $a,b \in \mathbb{Z}$. $\langle 1, 1+\sqrt{-5}\rangle$. So
$2 = (a + b\sqrt{5})(c+d\sqrt{-5}) \in \mathbb{Z}[\sqrt{-5}] $ and 
$1 + \sqrt{-5} = (a + b\sqrt{-5})(m+n\sqrt{-5})$. Take the norms to get 
\[ N(2) = 2^2 (a^2 + 5b^2)(c+5d^2) \]
\[ N(1+\sqrt{-5}) = (a^2+5b^2) (c^2 + 5d^2) \]
Thus
\[ (a^2+5b^2) \mid 4 \text{ and } (a^2+5b^2) \mid 6  \]
and so
\[ (a^2+5b^2) \mid 6 - 4 = 2 \]
Thus $(a^2+5b^2)  = 1, 2$, but if it is 1, then it is a unit. But there are no
solutions if it is 2. Thus no $a,b$ exist, and $\mathfrak{p}$ is not principal.
Claim 3: 
\[ \mathfrak{p}^2 = \langle 2 \rangle, \mathfrak{q}\frak{r} = \langle3\rangle,
\mathfrak{pq} = \langle 1 + \sqrt{-5} \rangle, \mathfrak{pr} = \langle 1 -
\sqrt{-5} \rangle \]
The upshot of this is that 
\[ \langle G \rangle = \langle 2 \rangle \langle 3 \rangle = \langle 1 +
\sqrt{-5} \rangle \langle 1 - \sqrt{-5} \rangle = p^2 qr = pqpr \]
Proof for $p^2 = \langle 2 \rangle$ is
\[ p^2 = \langle 2\cdot2 , 2(1+\sqrt{-5}), (1+\sqrt{-5})^2 \rangle \]
\[ p^2 = \langle 4, 2 + 2\sqrt{-5}, -4+2\sqrt{-5} +-(2+2\sqrt{-5}) \rangle \]
\[ p^2 = \langle 4, 2+2\sqrt{-5}, 6 \rangle \]
Since units don't matter and -6 can be replaced by 6. Furthermore we can use
$6-4 = 2$. Thus this is $\langle 2 \rangle $. We can add, subtract or multiply
be other generators.

\subsection{The Norm of an Ideal}
Consequences: 
\begin{enumerate}
\item Ideal GCD $\mathfrak{g}$ and LCM $\mathfrak{l}$. Let 
\[ \frak{a} = \prod_i \frak{p}_i^{e_i} \text{ and } \frak{b} = \prod_i
\frak{p}_i^{f_i} \]
GCD $\frak{g} | \frak{a},\frak{b}$. And if $\frak{g}'$ has the same properties
then $\frak{g}'|\frak{g}$. Rule
\[ \frak{g} = \prod_i \frak{p}_i^{\text{min}(e_i, f_i)}, \frak{l} = \prod_i
\frak{p}_i^{\text{max}(e_i, f_i)} \]
Lemma: 
\[ \frak{g} = \frak{a} + \frak{b} \]
\[ \frak{l} = \frak{a} \cap \frak{b} \]
\item We define the Norm of an ideal $\frak{N}$. 
\[ \frak{N}(\frak{a}) := |\mathcal{O} / \frak{a}| \]
Also, fyi,
\[ \Phi(\frak{a}) = |(\mathcal{O}/\frak{a})^*| \]
but we won't need this here. 

\end{enumerate}
Recall that for every nonzero ideal, $\frak{a}$ of $\mathcal{O}$ has a
$\mathbb{Z}$-basis $\{\alpha_1, \ldots, \alpha_n\}$ where $n = [K:\mathcal{O}]$.
Then 
\[ \frak{N}(\frak{a}) =
\left[\frac{\Delta[\alpha_1,\ldots,\alpha_n]}{\Delta_k}\right]^{1/2} \]
This is used in the homework.

A corollary, suppose that $\frak{a} = \langle \alpha \rangle$. Then
\[ \frak{N}(\frak{a}) = \left|N(\alpha)\right| \]
Proof, we let $\{\omega_1, \ldots, \omega_n\}$ be a $\mathbb{Z}$-basis for
$\mathcal{O}$. Then, $\{ \alpha \omega_1, \ldots \alpha \omega_n \}$ is a
$\mathbb{Z}$-basis for $\frak{a}$. Therefore, 
\[ \frak{N}(\frak{a}) = \left[ \frac{\Delta[\alpha \omega_1, \ldots, \alpha
\omega_n]}{\Delta[\omega_1, \ldots, \omega_n]} \right]^{1/2} =
\left[\frac{N(\alpha)^2 \Delta_k}{\Delta_k}\right]^2 = |N(\alpha)| \]

Some facts:
\[ \frak{N}(\frak{a}\frak{b}) = \frak{N}(\frak{a})\frak{N}(\frak{b}) \]
If $\frak{N}(\frak{a})$ is prime, then $\frak{a}$ is prime. 
Also, $\frak{N}(\frak{a}) \in \frak{a}$. This is because $\frak{N}(\frak{a}) =
|\mathcal{O}/\frak{a}|i \in \frak{a} $ by lagrange.
If $\frak{a}$ is prime then, $\frak{N}(\frak{a}) = p^m$ for some prime $p$ and
$m \leq [K:\mathbb{Q}]$. Theorem: finiteness results. 
\begin{enumerate}
\item Any nonzero ideal $\frak{a} \subseteq \mathcal{O}$ has a finite number of
divisors. 
\item Any nonzero ``rational integer'' in $\mathbb{Z}$ belongs to finitely many
ideals. This comes down to a norm calculation. 
\item Only finitely many ideals (false for numbers) of $\mathcal{O}$ have a
given fixed norm. This flows from the finiteness of the number of ideals in
$\mathcal{O}$. 
\end{enumerate}
Fact, let $\mathfrak{a}$ be an ideal. Then $\frak{a} = \langle \alpha, \beta
\rangle $ for some $\alpha, \beta \in \mathcal{O}$. At most 2 generators. For
example, $\mathbb{Z}[\sqrt{-5}]$. 
\[ \langle 6 \rangle = \frak{p}^2 \frak{q} \frak{r} \] as previously defined. 
Suppose $6 \in \frak{a}$. Then $\langle 6 \rangle \subseteq \frak{a}$, which is
equivalent to $\frak{a} | \langle 6 \rangle = p^2 q r $. 
So, $\frak{a} = \frak{p}^a \frak{q}^b \frak{r}^c, a\in \{0,1,2\}, b,c\in
\{0,1\}$. So 6 belongs to finitely many ideals. How many ideals have norm
$\langle 6 \rangle$. This can only happen when $\frak{a} | \langle 6 \rangle$ by
fact 3. Writing $\frak{a} = \frak{p}^a\frak{q}^b\frak{r}^c$, implies that
$N(\frak{a}) = 2^a 3^b3^c$ implies that $\frak{a} = \frak{pq}$ or $\frak{pf}$. 

Theorem: $\mathcal{O}$ factors into irreducible elements iff every ideal in
$\mathcal{O}$ is principle. First, PIDs are UIDs. Going the other way, it
suffices to show that every prime ideal is principal. Let $\frak{p}$ be a
nonzero prime in $\mathcal{O}$ Then $\frak{p} |
\langle\frak{N}(\frak{p})\rangle$. Note $n = \pi_1 \pi_2 \ldots \pi_s$. Since
$\frak{p}$ is prime, $\frak{p} | \langle \pi_i\rangle$ for some $i$. Since we
have factors in $\mathcal{O}$, $\pi_i$ is prime. Therefore $\langle
\pi_i\rangle$ is prime, and so $\frak{p} = \langle \pi_i\rangle$. 

Read/skim through chapters 6---8 from the book. 

\end{document}
